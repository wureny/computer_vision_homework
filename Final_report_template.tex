\documentclass[lettersize,journal]{IEEEtran}
\usepackage{amsmath,amsfonts}
\usepackage{algorithmic}
\usepackage{algorithm}
\usepackage{array}
\usepackage[caption=false,font=normalsize,labelfont=sf,textfont=sf]{subfig}
\usepackage{textcomp}
\usepackage{stfloats}
\usepackage{url}
\usepackage{verbatim}
\usepackage{graphicx}
\usepackage{cite}
\hyphenation{op-tical net-works semi-conduc-tor IEEE-Xplore}
% updated with editorial comments 8/9/2021

\begin{document}

\title{Your paper title}

\author{Li, 12234535; Zhang, 8980855; Wang 34346567  }  % <-this % stops a space

% The paper headers
\markboth{The final project report of computer vision, September 2025}%
{Shell \MakeLowercase{\textit{et al.}}: A Sample Article Using IEEEtran.cls for IEEE Journals}

\IEEEpubid{Macau University of Science and Technology, CS460/ EIE460/ SE460}
% Remember, if you use this you must call \IEEEpubidadjcol in the second
% column for its text to clear the IEEEpubid mark.

\maketitle

\begin{abstract}
Briefly introduce what this paper primarily accomplishes, the methods or models employed, the datasets used for validation, and the experimental results. Please include the project team’s GitHub URL in the last line of the abstract.
\end{abstract}

\begin{IEEEkeywords}
What key phrases best capture the main features of this paper?
\end{IEEEkeywords}

\section{Introduction}
Describe the task addressed in this work, its associated challenges and difficulties, the proposed solution, and highlight the novelty and unique aspects of your approach. Specify the datasets used for experiments and summarize the experimental outcomes.

\section{Related work}
outline the general methodologies adopted by prior studies, and clearly articulate the similarities and differences between existing approaches and your proposed method. 

\section{Technical Solution}
Provide a detailed description of your method/model, potentially including formulas, diagrams, or other illustrative elements.

\section{Experiments}
Introduce the experimental datasets, describe the experimental setup, and specify hyperparameter settings such as learning rate. Use quantitative and qualitative results, supported by tables and figures, to demonstrate performance.

\section{Discussion}
Analyze the strengths and limitations of the proposed method/model in light of the experimental results, and suggest potential directions for future improvement.

\section{Conclusion}
Please provide a summary of the entire paper, including what problem was addressed, what methodology was used, and what results were achieved.

\section{Description of member contributions}
Detail the specific contributions and responsibilities of each team member.

ATTENTION $\star \star \star $ Please list all references cited in the paper in the "References" section and ensure they are properly cited at the appropriate places within the text. $\star \star \star $
\begin{thebibliography}{1}
\bibliographystyle{IEEEtran}
\bibitem{ref1}
{\it{Mathematics Into Type}}. American Mathematical Society. [Online]. Available: https://www.ams.org/arc/styleguide/mit-2.pdf.
\bibitem{ref2}
H. Sira-Ramirez, ``On the sliding mode control of nonlinear systems,'' \textit{Syst. Control Lett.}, vol. 19, pp. 303--312, 1992.
\end{thebibliography}
\end{document}


